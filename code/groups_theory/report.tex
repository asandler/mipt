\documentclass[a4paper,12pt]{llncs}

\usepackage{amssymb,amsmath,amsthm}
\usepackage[utf8]{inputenc}
\usepackage[english, russian]{babel}
\usepackage{graphicx}
\usepackage{hyperref}
\usepackage{tikz}

\newcommand{\mypadding}{
    % rectangle 7x7 assuming page is a rectangle 9x9
    \textheight=231mm
    \textwidth=163.4mm
    \oddsidemargin=-2.1mm
    \evensidemargin=-2.1mm
    \topmargin=-5.5mm
}

\mypadding

\newcommand{\combus}[2]{\left(\begin{array}{c}#1 \\ #2 \end{array} \right)} % american style for C_n^k
\newcommand{\combru}[2]{C_{#1}^{#2}} % russian C_n^k
\newcommand{\comb}[2]{\combru{#1}{#2}}
\newcommand{\myN}{\mathbb{N}} % nice letters for common number sets
\newcommand{\myZ}{\mathbb{Z}}
\newcommand{\myR}{\mathbb{R}}
\newcommand{\myC}{\mathbb{C}}
\newcommand{\myQ}{\mathbb{Q}}
\newcommand{\myE}{\mathcal{E}} % basis
\newcommand{\myM}{\mathcal{M}} % some set
\newcommand{\mysetM}{\mathcal{M}}
\newcommand{\mysetN}{\mathcal{N}}
\newcommand{\walls}[1]{\left | #1 \right |} % |smth_vertically_large|
\newcommand{\pars}[1]{\left( #1 \right)} % (smth_vertically_large)
\newcommand{\class}[1]{\left[ #1 \right]} % [smth_vertically_large]
\newcommand{\bra}[1]{\langle #1 \rangle} % brackets for span of vectors, eg. <e_1, ..., e_k>
\newcommand{\myset}[1]{\left\{ #1 \right\}} % because { and } are special symbols in TeX
\newcommand{\mysetso}[2]{\myset{#1 \mid #2}}
%
% Example text:
% M = {x^2 | x is prime}
%
% Corresponding markup:
% $$ \mysetM = \mysetso{x^2}{x \text{ is prime}} $$
%
\newcommand{\myleq}{\leqslant}
\newcommand{\mygeq}{\geqslant}
\newcommand{\myempty}{\varnothing}
\newcommand{\myand}{\;\; \hm\& \;\;}				
\newcommand{\myor}{\; \hm\vee \;}					
\newcommand{\mytext}[1]{\mathop{\mathrm{#1}}}
\newcommand{\mychar}[1]{\mytext{char} #1} % characteristic of a field
\newcommand{\conj}[1]{\overline{#1}} % complex conjugation
\newcommand{\mycirc}{\circ}
\newcommand{\poly}[2]{#1 [ #2 ]} % ring of polynomials
\newcommand{\Rx}{\poly{\myR}{x}}
\newcommand{\Cx}{\poly{\myC}{x}}
\newcommand{\mydim}[1]{\dim #1}
\newcommand{\supp}[1]{\mytext{supp}(#1)}
\newcommand{\Sym}[1]{\mytext{Sym}(#1)}
\newcommand{\Alt}[1]{\mytext{Alt}(#1)}
\newcommand{\myfix}[1]{\mytext{Fix}(#1)}
\newcommand{\mysupp}[1]{\mytext{Supp}(#1)}
\newcommand{\myinv}[2]{\mytext{Inv}(#1, #2)}
\newcommand{\mycodim}[1]{\mytext{codim} #1}
\newcommand{\coords}[2]{\pars{#1}_{#2}}
\newcommand{\myrank}[1]{\mytext{rank} #1}
\newcommand{\myker}[1]{\mytext{Ker} #1}
\newcommand{\mysegment}[2]{[#1, #2]} 				
\newcommand{\myinterval}[2]{(#1, #2)} 				
\newcommand{\mypair}[2]{(#1, #2)}			
\newcommand{\myfunc}[3]{#1\!:\,#2 \hm\to #3} % TODO: make spaces between elements of this tag look better
\newcommand{\suchthat}{\!:\,}					
\newcommand{\subgroupindex}[2]{\walls{#1:#2}}
\newcommand{\myfactor}[2]{#1/#2}
\newcommand{\mycenter}[1]{Z(#1)}
\newcommand{\myisom}{\cong}
\newcommand{\mynormal}[2]{#1 \vartriangleleft #2}
\newcommand{\normalizer}[2]{N_{#1}(#2)}
\newcommand{\centralizer}[2]{C_{#1}(#2)}
\newcommand{\stabilizer}[2]{#1_{#2}}
\newcommand{\floor}[1]{\left \lfloor #1 \right \rfloor}

% Arrows
\newcommand{\myright}{\;\hm\Rightarrow\;}
\newcommand{\myleft}{\;\hm\Leftarrow\;}
\newcommand{\myleftright}{\;\hm\Leftrightarrow\;}
\newcommand{\myK}[2]{K_{#1, #2}}
\newcommand{\myG}[2]{G_{#1, #2}}
\newcommand{\mydiam}[2]{d(#1, #2)}

\newcommand{\somehow}{справа }
\newcommand{\notsomehow}{слева }

\newtheorem{mytheorem}{Теорема}

\newtheorem*{myhypothesis}{Гипотеза}
\newtheorem{mytheorem}{Теорема}
\newtheorem*{mylem}{Лемма}


\begin{document}

\title{Сортировка на $(h, 3)$-призме}
\author{Биктимиров Айдар, Сандлер Андрей}
\institute{Московский физико-технический институт (государственный университет)\\
Факультет инноваций и высоких технологий\\
\[\]
Научный руководитель: Беляев Виссарион Викторович, д.ф.-м. н., проф.}

\maketitle

\begin{abstract}

В работе рассматривается задача сортировки занумерованных вершин графа специального вида,
причем при сортировке разрешается менять местами только смежные вершины. Для графов небольших
размеров приводится результат компьютерных вычислений и точный ответ, для больших ситуаций
доказываются некоторые оценки и выдвигаются гипотезы относительно самых удаленных друг от друга нумераций.

\end{abstract}

\[\]

\begin{center}
\begin{tikzpicture}[scale=1, thick]
\node [fill=black, shape=circle, inner sep=1.4pt] (A) at (0,0) {};
\node [fill=black, shape=circle, inner sep=1.4pt] (B) at (2,0) {};
\node [fill=black, shape=circle, inner sep=1.4pt] (C) at (4,0) {};
\node [fill=black, shape=circle, inner sep=1.4pt] (D) at (6,0) {};
\node [fill=black, shape=circle, inner sep=1.4pt] (E) at (1,1) {};
\node [fill=black, shape=circle, inner sep=1.4pt] (F) at (3,1) {};
\node [fill=black, shape=circle, inner sep=1.4pt] (G) at (5,1) {};
\node [fill=black, shape=circle, inner sep=1.4pt] (H) at (7,1) {};
\node [fill=black, shape=circle, inner sep=1.4pt] (I) at (0,2) {};
\node [fill=black, shape=circle, inner sep=1.4pt] (J) at (2,2) {};
\node [fill=black, shape=circle, inner sep=1.4pt] (K) at (4,2) {};
\node [fill=black, shape=circle, inner sep=1.4pt] (L) at (6,2) {};

\draw (A) -- (B) -- (C) -- (D);
\draw (E) -- (F) -- (G) -- (H);
\draw (I) -- (J) -- (K) -- (L);
\draw (A) -- (E) -- (I) -- (A);
\draw (B) -- (F) -- (J) -- (B);
\draw (C) -- (G) -- (K) -- (C);
\draw (D) -- (H) -- (L) -- (D);


\end{tikzpicture}
\end{center}

\[\]
\[\]

\section{Введение}

Сортировка на графе с помощью множества разрешенных транспозиций -- в общем случае ещё не решенная задача. Звучит она так:
\begin{quote}
\textit{
пусть на вершинах графа задана некоторая нумерация натуральными числами от 1 до n. Какое минимальное число перестановок смежных
вершин необходимо сделать в этом графе, чтобы получить некоторую другую нумерацию?
}
\end{quote}
Ставится также вопрос:
\begin{quote}
\textit{
какие две нумерации вершин данного графа наиболее удалены друг от друга, то есть сводятся одна к другой за максимальное число шагов?
}
\end{quote}

В данной работе мы рассматриваем частный случай этой задачи - сортировку на одном конкретном графе и его естественном обобщении. Для небольших ситуаций мы получили точный ответ на компьютере, а в общем случае оценили ответ сверху и снизу достаточно близко.

\newpage
\section{Терминология}

Изначальный рассматриваемый нами граф выглядит следующим образом:

\begin{center}
\begin{tikzpicture}[scale=1, thick]
\node [fill=black, shape=circle, inner sep=1.4pt] (A) at (0,0) {};
\node [fill=black, shape=circle, inner sep=1.4pt] (B) at (0,2) {};
\node [fill=black, shape=circle, inner sep=1.4pt] (C) at (1,1) {};
\node [fill=black, shape=circle, inner sep=1.4pt] (D) at (2,0) {};
\node [fill=black, shape=circle, inner sep=1.4pt] (E) at (2,2) {};
\node [fill=black, shape=circle, inner sep=1.4pt] (F) at (3,1) {};

\draw (A) -- (D) -- (F) -- (E) -- (B) -- (A);
\draw (A) -- (C) -- (B);
\draw (D) -- (F) -- (E);
\draw (C) -- (F);
\draw (D) -- (E);

\end{tikzpicture}
\end{center}

	Это призма с тремя вершинами на каждом из двух слоев. Два возможных варианта его обобщения - наращивать число вершин на уровне
и наращивать число уровней. Отсюда название графа -- $(h, c)$-призма -- призма с $h$ слоями и $c$ вершинами на каждом слое. Мы выбрали второй
вариант обобщения. Таким образом, данная работа посвящена исследованию $(h, 3)$-призмы.

	Обозначим через $n$ число вершин в графе. Заметим, что всевозможные нумерации вершин образуют симметрическую группу $S_{n}$.
Присвоим каждой нумерации (перестановке) номер, равный её номеру в лексикографическом порядке на подстановках. Теперь рассмотрим
некоторый подграф графа Кэли, построенный по группе подстановок и множеству ребер графа, считая их множеством разрешенных транспозиций
в $S_{n}$. Будем строить его послойно: на нулевой слой поместим одну вершину и дадим ей номер исходной перестановки. На первый слой поместим
вершины с номерами тех перестановок, которые получаются из исходной с помощью ровно одной транспозиции, и так далее. Выражаясь языком Computer
Science, запустим обход в ширину на графе Кэли и возьмем в качестве подграфа дерево обхода. Номер последнего слоя в полученном дереве и будет
ответом на второй вопрос задачи.


Также будет удобно называть количество транспозиций, необходимых для перевода данной нумерации в начальную, длиной этой нумерации (или перестановки),
а наибольшую возможную длину нумерации -- диаметром призмы (графа).

\section {Компьютерный эксперимент}

Ответ на задачу при небольших значениях $h$ мы решили посчитать на компьютере. Для определения диаметра графа Кэли использовался алгоритм
поиска в ширину. В процессе его работы требуется хранить очередь из перестановок, а также сохранять список уже посещенных вершин, чтобы поиск мог
завершиться, а не ушел в бесконечный цикл.

	Первоначальная реализация не была оптимизирована, требовала $C \cdot n \cdot n!$ памяти, где $C > 1$, а также была довольно медленной
(сложность работы алгоритма составляла $O(n! \cdot log(n!) \cdot n)$). В результате мы смогли проанализировать только $(2, 3)$- и $(3, 3)$-призмы.
По результатам первого эксперимента мы выдвинули гипотезу о том, что все нумерации, которые получаются из начальной перестановкой слоев в обратном
порядке и поворотом этих слоев, будут иметь максимальную длину.

	Далее мы оптимизировали алгоритм путем преобразования самих перестановок в их номера. В результате этого сложность алгоритма снизилась
до $O(n! \cdot log(n!))$, а потребление памяти - до $(n! / 8)$, благодаря чему стал возможен анализ $(4, 3)$-призмы. Это опровергло нашу первую
гипотезу о виде перестановок наибольшей длины.

	Кроме определения диаметра графа Кэли, мы добавили в программу возможность полного вывода его остовного дерева, что позволяет определить для
каждой из перестановок один из минимальных по длине способов привести ее к начальной.


\subsection{Послойная функция роста}

Послойной функцией роста (для нашего графа) называется функция натурального аргумента, которая показывает, сколько на нём существует нумераций заданной
длины.

	Ниже приведены результаты вычисления диаметра и послойной функции роста графа Кэли для (2, 3)-, (3, 3)- и (4, 3)-призм. Номер строки указывает
длину перестановки, число справа -- количество перестановок такой длины.

\begin{table}
\caption{Значения послойной функции роста при $n = 6, 9$ и $12$.}
\begin{center}
\begin{tabular*}{0.25\textwidth}{@{\extracolsep{\fill}}|c|c|c|c|}
\hline
  & $n = 6$ & $n = 9$ & $n = 12$ \\
\hline
1 & 9 & 15 & 21 \\
\hline
2 & 46 & 129 & 248 \\
\hline
3 & 148 & 788 & 2103 \\
\hline
4 & 253 & 3615 & 13950 \\
\hline
5 & 199 & 12729 & 75525 \\
\hline
6 & 60 & 33856 & 340471 \\
\hline
7 & 4 & 65895 & 1289604 \\
\hline
8 & & 91275 & 4116728 \\
\hline
9 & & 85052 & 11058204 \\
\hline
10 & & 50217 & 24894822 \\
\hline
11 & & 16903 & 46667348 \\
\hline
12 & & 2347 & 72221112 \\
\hline
13 & & 58 & 91109874 \\
\hline
14 & & & 91795393 \\
\hline
15 & & & 71807347 \\
\hline
16 & & & 41874833 \\
\hline
17 & & & 16969822 \\
\hline
18 & & & 4219024 \\
\hline
19 & & & 520390 \\
\hline
20 & & & 24204 \\
\hline
21 & & & 562 \\
\hline
22 & & & 14 \\
\hline
\end{tabular*}
\end{center}
\end{table}


Таким образом, мы получили ответ на вопрос задачи для призм с параметром $h < 5$. Диаметр $(2, 3)$-призмы равен семи,
диаметр $(3, 3)$-призмы равен тринадцати, а диаметр $(4, 3)$-призмы -- двадцати двум.

\section {Оценка снизу через подсчет девиаций}

Основываясь на данных компьютерного эксперимента, мы выдвинули ещё одну гипотезу о том, как выглядят две самых удаленных друг от друга перестановки.
\[\]
\textbf{Гипотеза.} Одна из нумераций $(h, 3)$-призмы с самой большой длиной получается из начальной перестановкой слоев призмы в обратном порядке
и поворотом всех слоев на единицу в одну сторону.
\[\]
	

	Используя эту гипотезу, мы приводим нижнюю оценку диаметра графа Кэли $(h, 3)$-призмы. Для этого потребуется ввести одно


\[\]
\textbf{Определение.} Назовем \textit{девиацией (или отклонением)} вершины $(h, 3)$-призмы минимальное число транспозиций, необходимое для того, чтобы
переместить эту вершину на её место в исходной перестановке.
\[\]

	Тогда в определенных выше терминах верна

\[\]
\begin{flushleft}
\textbf{Теорема.} Диаметр графа Кэли $(h, 3)$-призмы не меньше $\displaystyle \frac{n^2}{12} + C$, где $C = 0$ при четных $n$ и $\displaystyle C = -\frac{3}{2}$ при нечетных $n$.
\end{flushleft}
\textit{Доказательство.} Для доказательства этой оценки достаточно найти одну такую нумерацию $(h, 3)$-призмы, которая не сортируется быстрее, чем
за указанное в условии число шагов.
	Рассмотрим нумерацию из гипотезы и посчитаем сумму $S$ девиаций во всех вершинах призмы.
\[
	S = 3 \cdot \sum_{i=1}^{h} \left ( 2 \cdot \left | \frac{h}{2} - i \right | \right )
\]

	Обозначим через $S_{even}$ сумму $S$ при четных $n$, а через $S_{odd}$ - сумму $S$ при нечетных $n$. Путем несложных вычислений получаем, что
$\displaystyle S_{even} = \frac{n^2}{6} $, а $\displaystyle S_{odd} = \frac{n^2}{6} - \frac{3}{2}$.

	Заметим, что каждая транспозиция либо не изменяет $S$, либо увеличивает/уменьшает это значение на $2$. Но в исходной нумерации сумма
девиаций равна нулю, значит, каким бы способом мы не сводили нашу перестановку к исходной, нам нужно получить нулевую сумму девиаций, что невозможно
меньше, чем за $\displaystyle \frac{S}{2} = \frac{n^2}{12} + C$ шагов. Таким образом, мы доказали теорему.

\section {Оценка сверху при помощи алгоритма}

Для получения верхней оценки ответа на задачу мы приведем алгоритм, который переводит любую перестановку к исходной, а также покажем время его работы.

	Данный алгоритм похож на сортировку выбором: на каждом шаге ищется минимальная по номеру вершина графа, стоящая не на своем месте, после чего она
поднимается на нужный слой. Для этого вершина переставляется в тот ряд (всего их в призме три), в котором она должна находиться (за 0 либо 1 транспозицию),
и далее несколькими транспозициями поднимается на свое место, как в сортировке пузырьком.

	Ясно, что наибольшее кол-во транспозиций для поднятия вершины на свое место требуется, если она <<всплывает>> с нижнего слоя призмы. Также заметим, что
если k верхних слоев призмы отсортированы правильно, то любую из оставшихся вершин можно вернуть на место не больше чем за $(h - k)$ транспозиций.

	Смоделируем наихудшую для нашего алгоритма ситуацию, когда для поднятия каждой вершины требуется вернуть ее на нужный ряд и применить $\displaystyle (h - \frac{i}{3})$
транспозиции для <<всплытия>>. Суммарное число транспозиций, которое придется выполнить, равно
\[
3 \cdot \sum_{i=1}^{h} (h - i) + n = 3 \cdot \sum_{i=1}^{h}i + n = 3 \cdot \frac{h \cdot (h + 1)}{2} + n = \frac{n^2}{6} + \frac{3 \cdot n}{2}.
\]

	Таким образом, мы доказали, что любая нумерация вершин сводится к исходной не больше, чем за указанное количество транспозиций.

\section{Заключение}
В своей работе мы ввели понятие $(h, c)$-призмы и рассмотрели задачу о сортировке на $(h, 3)$-призме. Для $h > 4$ ответ на задачу ещё не получен,
но мы привели его оценки сверху и снизу, которые можно продолжать улучшать, так как они различаются асимптотически в два раза. Для $h \leqslant 4$
точный ответ удалось посчитать на компьютере.

\begin{thebibliography}{99}
\bibitem{bel-tg}Беляев В.\,В. Лекции по алгебре, геометрии и теории групп.
\texttt{https://sites.google.com/site/miptmath} --- 2010--2011 гг.
\end{thebibliography}

\end{document}
